\documentclass[]{article}
\usepackage[utf8]{inputenc}
\usepackage{amsthm}
\usepackage{amsfonts}

\newtheorem{mydef}{Définition}
\newtheorem{mythm}{Théorème}
\newtheorem{mycoro}{Corollaire}

\newcounter{ppc}
\newcounter{pspc}
\newcounter{psspc}

\newcommand{\proofpart}[1]{
	\bigskip
	\\
	\setcounter{pspc}{1}
	\underline{\textbf{\Roman{ppc}-#1}}
	\stepcounter{ppc}
}

\newcommand{\proofsubpart}[1]{
	\medskip
	\\
	\setcounter{psspc}{1}
	\arabic{pspc}{)} \underline{#1}
	\stepcounter{pspc}
}

\newcommand{\proofsubsubpart}[1]{
	\\
	\textit{\alph{psspc}{)} \underline{#1}}
	\stepcounter{psspc}
}

\newenvironment{myproof}[1]{
	\noindent
	\textit{Preuve}
	\ifx&#1&
	\else
		: #1
	\fi
	\\
	\setcounter{ppc}{1}
}

\begin{document}

\begin{myproof}{Théorème de Lagrange}
	Soit $G$ un groupe, et $H$ un sous-groupe de $G$.\\
	\indent On va démontrer que $\#H \mid \#G$ en créant une partition de $G$, dont chaque partie aura le même cardinal que $H$.
	\proofpart{Relation d'équivalence $\sim$ sur $G$}
	\proofsubpart{Définition de la relation}
	\\
	Soit $\sim$ la relation telle que : $\forall x, y\in G,~x \sim y \Leftrightarrow x^{-1}y \in H \Leftrightarrow \exists h \in H : xh=y$
	\proofsubsubpart{$\sim $ est réflexive :}\\
	$x \sim x, x \in G$ car $x^{-1}x=e \in H$
	\proofsubsubpart{$\sim $ est symétrique :}\\
	$x \sim y \Leftrightarrow x^{-1}y \in H \Leftrightarrow (x^{-1}y)^{-1}=y^{-1}x \in H \Leftrightarrow y \sim x$
	\proofsubsubpart{$\sim $ est transitive :}\\
	$
	\left.
	\begin{array}{cc}
		x \sim y \Leftrightarrow x^{-1}y \in H \\
		x \sim z \Leftrightarrow x^{-1}z \in H
	\end{array}
	\right\}
	\Rightarrow (x^{-1}y)(y^{-1}z)=x^{-1}(yy^{-1})z=x^{-1}z \in H \Leftrightarrow x\sim z
	$
	\proofsubpart{Classes d'équivalence}\\
	Soit $x \in G$.\\
	\indent $xH$ est la classe d'équivalence égale à $\{y \in G | x \sim y\}$, on peut ainsi partitionner $G$ en un ensemble de classes d'équivalence $(x_iH)_{i \in I}$, on à donc :\\
	$\displaystyle G = \coprod_{i \in I}x_iH$ d'où l'égalité $\displaystyle \#G=\sum_{i \in I}\#(x_iH)$
	\\
	\underline{N.B. :} L'ensemble des classes que l'on vient de définir se note $G/H$.
	\proofsubpart{Equipotence entre $H$ et $xH$}\\
	Soit $x \in G$.\\
	\indent $xH = \{y \in G | x \sim y\} = \{y \in G | \exists h \in H : y = xh\}$. On a donc l'application 
	$\varphi :
	\begin{array}{cc}
		H \to xH \\
		h \mapsto xh
	\end{array}$ qui est bijective.
	\proofsubsubpart{Surjectivité :} $xH = \{xh | h \in H\} = \varphi(H)$
	\proofsubsubpart{Injectivité :} $h, h' \in H : \varphi(h)=\varphi(h') \Rightarrow xh=xh' \Rightarrow h=h'$ en multipliant par $x^{-1}$ à gauche.\\
	Ainsi, $\varphi$ est bijective, on en déduit $\#H=\#(xH)$
	\proofpart{Divisibilité}
	\\
	\indent On reprend l'égalité $\displaystyle \#G=\sum_{i \in I}\#(x_iH)$, on sait à présent que pour tout $x$ de $G$ on a $\#H=\#(xH)$, donc on a bien $\#H|\#G$.
\end{myproof}
\end{document}